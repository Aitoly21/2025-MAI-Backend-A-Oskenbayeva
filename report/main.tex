\documentclass[a4paper,12pt]{article}

\usepackage[utf8]{inputenc}
\usepackage[T2A]{fontenc}
\usepackage[russian]{babel}
\usepackage{geometry}
\usepackage{enumitem}
\usepackage{csquotes}
\usepackage{setspace}
\usepackage{float}
\usepackage{graphicx}



\geometry{margin=2.5cm}
\setlength{\parindent}{0pt}
\linespread{1.2}

\begin{document}

% Титульная страница
\begin{titlepage}
\begin{center}
\bfseries
{\Large Московский авиационный институт\\
(национальный исследовательский университет)}

\vspace{48pt}
{\large Факультет информационных технологий и прикладной математики}

\vspace{36pt}
{\large Кафедра вычислительной математики и программирования}

\vspace{48pt}
Курсовой проект по курсу «Программная инженерия»

\end{center}

\vspace{72pt}
\begin{flushright}
\begin{tabular}{rl}
Студент: & А.\,У. Оскенбаева \\
Преподаватель: & А.\,А. Кухтичев \\
Группа: & М80-207М-23 \\
Дата: & \\
Оценка: & \\
Подпись: & \\
\end{tabular}
\end{flushright}

\vfill
\begin{center}
\bfseries
Москва, \the\year
\end{center}
\end{titlepage}

\newpage

% Задание курсового проекта
\section*{Задание курсового проекта}

\textbf{Задача:} Требуется разработать современное веб-приложение. Основные требования:

\begin{itemize}[leftmargin=1.2cm]
  \item Поднять веб-сервер (nginx). Отдавать статические файлы (логотип и т.д.) по \texttt{/static/}. Настроить проксирование запросов на сервер-приложение по отдельному URL;

  \item В конфиге nginx создать \texttt{location}, которое будет ходить на Django-приложение;

  \item Поднять сервер-приложение;

  \item Создать базу данных в PostgreSQL; Написать классы-модели, мигрировать;

  \item Организовать приём и передачу сообщений с помощью формата JSON, используя REST;

  \item Реализовать метод API для загрузки файла, использовать для хранения файла облачное S3 хранилище, создать \texttt{location} в Nginx для раздачи загруженных файлов, реализовать обработчик в приложении для проверки прав доступа к файлу;

  \item Реализовать OAuth2-авторизацию для двух любых социальных сетей, навесить декоратор, проверяющий авторизацию при вызовах API;

  \item Покрыть тестами все вьюхи и по желанию другие функции; Написать selenium-тест (найти элемент + клик на элемент); Использовать mock-объект при тестировании; Использовать factory boy; Узнать степень покрытия тестами с помощью библиотеки coverage;

  \item Развернуть и наполнить тестовыми данными Elasticsearch; Реализовать поиск по пользователям, продуктам (сущностям); Реализовать метод API для поиска по указанным сущностям и создать страницу HTML с вёрсткой для поиска и отображения результатов;

  \item Установить и поднять centrifugo; Подключить centrifugo к проекту на стороне клиента и сервера; Организовать отправку/получение сообщений с помощью centrifugo;

  \item Установить docker и docker-compose; Создание Dockerfile для Django-приложения; Создание docker-compose для проекта:
  \begin{itemize}
      \item nginx;
      \item База данных;
    \end{itemize}  % конец предыдущего списка

\newpage  % ← новая страница начинается прямо перед следующими пунктами
\begin{itemize}
    \item Django-приложение;
    \item elasticsearch;
    \item Создание Makefile для проекта.
\end{itemize}
\vspace{1.5\baselineskip}
\textbf{Тема курсовой:} Онлайн платформа Beauty Academy.

\textbf{Вариант веб-сервера:} nginx. \\
\textbf{Вариант сервера-приложений:} Django. \\
\textbf{Вариант базы данных:} PostgreSQL.

\newpage  % ← начинает новую страницу

\section{Веб-сервер}

В проекте используется \textbf{Nginx} в качестве веб-сервера и обратного прокси для Django-приложения Beauty Academy.

\textbf{Конфигурация включает:}

\begin{itemize}
    \item Проксирование API-запросов к Django-приложению:
    \begin{itemize}
        \item Все запросы к \texttt{/api/} направляются на порт 8000, где работает \textbf{runserver}.
        \item Настроен proxy-pass в конфиге nginx.
    \end{itemize}

    \item Обслуживание статики:
    \begin{itemize}
        \item HTML и изображения отдаются через \texttt{/web/} с директории \texttt{public/}.
    \end{itemize}

    \item Разделение логики через Nginx:
    \begin{itemize}
        \item \texttt{/api/} — для API-запросов.
        \item \texttt{/web/} — для HTML и фронтенда.
    \end{itemize}

    \item Сервер слушает на порту 80, проксирует на порт 8000.
    \item Проверка API через \texttt{curl} и работу через Gunicorn + Nginx.
\end{itemize}

\vspace{1em}
\begin{figure}[H]
    \centering
    \includegraphics[width=0.85\linewidth]{photo_2025-05-08_22-56-45.jpg}
    \caption{Пример моделей Django для проекта Beauty Academy}
    \label{fig:nginx-config}
\end{figure}

\section{Сервер-приложений}

Модели данных:

\begin{itemize}
    \item \textbf{Users} — информация о пользователях (логин, пароль, роль, дата рождения, номер телефона, адрес, и т.д.)
    \item \textbf{Courses} — курсы на платформе (название, описание, категория, режим, длительность)
    \item \textbf{Categories} — категории курсов (макияж, уход за кожей, брови и т.д.)
\end{itemize}

\vspace{1em}
\noindent \textbf{}

\begin{figure}[H]
    \centering
    \includegraphics[width=0.85\linewidth]{photo_2025-05-09_01-16-53.jpg}
    \caption{Models.py}
    \label{fig:nginx-config}
\end{figure}

\vspace{1em}
\textbf{Эндпоинты API:}

\begin{itemize}
    \item \textbf{/api/profile/} — информация о пользователе (GET, POST)
    \item \textbf{/api/courses/} — список курсов (GET, POST)
    \item \textbf{/api/categories/} — категории курсов (GET, POST)
    \item \textbf{Поиск:} \texttt{/api/profile/?q=студент}
\end{itemize}

\noindent \textbf{}
\begin{figure}[H]
    \centering
    \includegraphics[width=0.85\linewidth]{photo_2025-05-09_01-20-51.jpg}
    \caption{Маршруты API в файле \texttt{urls.py}}
    \label{fig:nginx-config}
\end{figure}

\section{База данных и ORM}

\textbf{Конфигурация:}

\begin{itemize}
    \item Используется PostgreSQL 15.
    \item Подключение настроено в \texttt{settings.py}:
    \begin{itemize}
        \item \texttt{ENGINE = 'django.db.backends.postgresql'}
        \item \texttt{HOST = 'db'}
        \item \texttt{USER = 'beauty_user'}
        \item \texttt{NAME = 'beautyacademy_db'}
    \end{itemize}
    \item Подключение проверено через \texttt{psql}, команды \verb|\l| и \verb|\du|.
    \item Суперпользователь: \texttt{admin / admin12345}
\end{itemize}

\textbf{Миграции:}

\begin{enumerate}
    \item Первая миграция создаёт таблицы всех моделей:
    \begin{itemize}
        \item \texttt{Profile}, \texttt{Course}, \texttt{Category}, \texttt{CourseSchedule}, \texttt{Attendance}
    \end{itemize}

    \item Проверка миграций:
    \begin{itemize}
        \item \texttt{python manage.py makemigrations}
        \item \texttt{python manage.py migrate}
    \end{itemize}

    \item Проверка через PostgreSQL:
    \begin{itemize}
        \item \texttt{sudo -u postgres psql -d beautyacademy_db}
        \item \verb|\dt| — показать все таблицы
        \item \verb|SELECT * FROM core_profile LIMIT 5;|
    \end{itemize}
\end{enumerate}

\textbf{Генерация тестовых данных:}

\begin{itemize}
    \item Используется через Django-админку: \texttt{/admin/}
    \item Данные вводились вручную (имя, курс, категория)
    \item Проверено отображение всех моделей в админке:
    \begin{itemize}
        \item \texttt{Profile}, \texttt{Course}, \texttt{Category}, \texttt{CourseSchedule}
    \end{itemize}
    \item Использовались реальные названия категорий (например: Макияж, Брови)
\end{itemize}

\begin{figure}[H]
    \centering
    \includegraphics[width=0.85\linewidth]{photo_2025-05-09_01-34-24.jpg}
    \label{fig:nginx-config}
\end{figure}

\begin{figure}[H]
    \centering
    \includegraphics[width=0.85\linewidth]{photo_2025-05-09_01-34-41.jpg}
    \caption{Миграции и тестовые данные в файле \texttt{admin.py}}
    \label{fig:nginx-config}
\end{figure}

\section{Контейнеризация}

Проект развёрнут с использованием \textbf{Docker} и \textbf{Docker Compose}.

\vspace{1em}
\textbf{Сервисы:}

\begin{enumerate}
    \item \textbf{db (PostgreSQL)}
    \begin{itemize}
        \item Используется официальный образ \texttt{postgres:15}
        \item Переменные окружения задаются в \texttt{docker-compose.yml}:
        \begin{itemize}
            \item \texttt{POSTGRES_DB=beautyacademy\_db}
            \item \texttt{POSTGRES_USER=beauty\_user}
            \item \texttt{POSTGRES_PASSWORD=beauty\_pass}
        \end{itemize}
        \item Данные сохраняются в том \texttt{pg\_data}
        \item Настроен \texttt{healthcheck} для проверки доступности сервиса
    \end{itemize}

    \item \textbf{web (Django)}
    \begin{itemize}
        \item Собирается с помощью \texttt{Dockerfile}
        \item При старте выполняются миграции и создаётся суперпользователь
        \item Подключается к БД через параметры окружения
        \item Зависит от PostgreSQL (через \texttt{depends\_on})
        \item Настроен \texttt{healthcheck}
    \end{itemize}

    \item \textbf{nginx}
    \begin{itemize}
        \item Используется официальный образ \texttt{nginx:latest}
        \item Проксирует запросы:
        \begin{itemize}
            \item \texttt{/api/} \textrightarrow{} web:8000
            \item \texttt{/web/} \textrightarrow{} директория \texttt{public/}
        \end{itemize}
        \item Монтируются конфиги и статика
        \item Зависит от web-приложения
        \item Настроен \texttt{healthcheck}
    \end{itemize}
\end{enumerate}

\vspace{1em}
\noindent\textbf{} 
\begin{figure}[H]
    \centering
    \includegraphics[width=0.85\linewidth]{photo_2025-05-09_01-40-07.jpg}
    \caption{Фрагмент файла \texttt{docker-compose.yml}, содержащий конфигурацию сервисов базы данных PostgreSQL}
    \label{fig:nginx-config}
\end{figure}

\newpage
\section{Выводы}

В рамках реализации проекта Beauty Academy я научилась разворачивать полноценное веб-приложение с использованием современных технологий. Было освоено построение структуры Django-проекта, реализация моделей, API и связей между сущностями. Я закрепила навыки работы с PostgreSQL, настройки миграций и админ-панели, а также научилась использовать Docker и docker-compose для упрощения запуска и деплоя.

Особое внимание было уделено настройке Nginx как обратного прокси и разделению маршрутов по функциональности. Я научилась реализовывать взаимодействие между сервисами, настраивать зависимости и healthchecks. Полученные знания пригодятся мне как при разработке собственных проектов, так и при работе в команде над большими системами.

\newpage
\section*{Список литературы}

\begin{enumerate}
    \item \textit{Официальная документация Django} \\
    URL: \texttt{https://docs.djangoproject.com/en/5.2/} (дата обращения: 25.04.2025).

    \item \textit{Официальная документация Docker Compose} \\
    URL: \texttt{https://docs.docker.com/compose/} (дата обращения: 01.05.2025).

    \item \textit{Официальная документация Selenium WebDriver} URL: \texttt{https://www.selenium.dev/documentation/} (дата ораени: 02.05.2025).

    \item \textit{Официальная документация NGINX} \\
    URL: \texttt{https://nginx.org/en/docs/} (дата обращения: 15.04.2025).
\end{enumerate}

\end{document}
